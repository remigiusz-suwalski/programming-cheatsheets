\section{Shell style guide}
The following notes are meant to be summary of a style guide written by Paul Armstrong and too many more to mention (revision 1.26).

Bash is the only shell scripting language permitted for executables.
Bash should only be used for simple wrapper scripts or small utilities.

Executables should have no extension, libraries must have a \texttt{.sh} extension and should not be executable.
SUID and SGID are \emph{forbidden} on shell scripts.

All error messages should go to \texttt{STDERR},
a function to print out error messages along with other status information is recommended:
\begin{minted}{bash}
err () {
  echo "[$(date +%Y-%m-%d\ %T)]: $@" >&2
}
\end{minted}

\textbf{Comments}.
Start each file with a description of its contents.
Any function that is in a library or not both obvious and short, must be commented.
Comment tricky, interesting or important parts of code.
Use TODO comments for temporary or good enough but not perfect code and short-term solutions.

The following are required for any new code:
\begin{compactenum}
\item Indent 2 spaces, no tabs.
\item Maximum line length is 80 characters.
\item Long pipelines should be split one per line. % if they don't all fit on one line.
% \item Put \texttt{; do} and \texttt{; then} on the same line as the \texttt{while}, \texttt{for} or \texttt{if}.
\item Indent case alternatives by 2 spaces.
\item Always quote strings containing variables, command substitutions or spaces. %or shell meta characters. % , unless unquoted expansion is required. 
\end{compactenum}

Use quotes rather than filler characters if possible.
Use an explicit path when doing wildcard expansion of filenames.
Avoid \texttt{eval}.
Use process substitution or for loops in preference to piping to while.
Finally, \texttt{[[ ... ]]} is preferred over \texttt{[} and \texttt{test}.

\textbf{Naming conventions}.
Function and variable names should be lower case, with underscore to separate words.
Constants and environment variable names should be all caps, declared at the top of the file.
Use \texttt{readonly} or \texttt{declare -r} to ensure they're read only.
Declare function specific variables with \texttt{local}.
A function called main is required for scripts long enough to contain at least one other function.

\subsection{Other useful resources}
If you want to understand Bash better, consider one of the following webpages:
\begin{compactenum}
\item \url{http://wiki.bash-hackers.org}
\item \url{http://mywiki.wooledge.org/BashGuide}
\end{compactenum}

\textbf{Avoid} guides by TLDP (The Linux Documentation Project) at all costs.
