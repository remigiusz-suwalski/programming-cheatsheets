\documentclass[a4paper, twoside, 8pt]{extarticle}
\usepackage[
    left=1.2cm,
    right=1.2cm,
    top=2.25cm,
    bottom=1.25cm]{geometry}
\usepackage{multicol}

% extensive control of page headers and footers
\usepackage{fancyhdr} 
\makeatletter
\fancypagestyle{mypagestyle}
{\newpage \fancyfoot[C]{} \renewcommand{\footrulewidth}{0pt}}
\makeatother
\pagestyle{mypagestyle}
\headsep 5pt            

% highlighted source code
\usepackage{minted} 
\usemintedstyle{pastie}

% selects alternative section titles
\usepackage[compact]{titlesec} 
\titlespacing{\section}{0pt}{\parskip}{0}
\titlespacing{\subsection}{0pt}{\parskip}{0}
\titlespacing{\subsubsection}{0pt}{\parskip}{0}

% custom font
\usepackage{Alegreya}
% provides an extended symbol collection
\usepackage{amssymb} 
\usepackage{hyperref}
% reference last page
\usepackage{lastpage}
% layout with zero \parindent, non-zero \parskip 
\usepackage{parskip} 
% driver-independent color extensions
\usepackage{xcolor} 
% compact list environment
\usepackage{paralist}
\usepackage{microtype}

\newcommand{\manualbreak}{\vspace*{\fill}\columnbreak}

\newcommand{\cmdcore}{$\blacksquare$}
\newcommand{\cmdutil}{$\boxplus$}
\newcommand{\cmdvar}{$\square$}
 
\usepackage[utf8]{inputenc}
\usepackage[T1]{fontenc}

\begin{document}
\renewcommand{\footrulewidth}{0.4pt}
\fancyhead[LE,LO]{Linux/Bash -- quick reference (page \thepage/\pageref{LastPage})}
\fancyhead[RO,RE]{source: \url{https://git.io/vMQlp}} 
\fancyfoot[RF]{author: Remigiusz Suwalski,\\ date: \today}
\fancyfoot[LF]{\cmdcore \,: coreutils 8.27, \cmdutil \,: util-linux 2.29, \cmdvar \,: other commands.}

\begin{multicols*}{3}
\textbf{Bash}, a command line interface for interacting with the operating system, was created in the 1980s.
Other popular shells are \emph{zsh} and \emph{fish}.

\section{Programming in Bash}
\subsection{Shebang}
The shebang (\texttt{\#!}) at the head of a script indicates an 
interpreter for execution, as in \texttt{\#!/bin/bash}.
Lines starting with a \texttt{\#} (with the exception of shebang) 
are comments and thus won't be executed.


\section{Emacs shortcuts in Bash}
For detailed information on emacs editing mode, visit 
\url{http://readline.kablamo.org/emacs.html} or type \texttt{man readline}.

\begin{compactenum}
\item [\texttt{ Ctrl-l}] clears the screen. 
\item [] 
\item [\texttt{ Ctrl-a}] moves to the start of the line,
\item [\texttt{ Ctrl-e}] moves to the end of the line. 
\item []
\item [\texttt{ Ctrl-u}] deletes to the beginning of the line,
\item [\texttt{ Ctrl-k}] deletes to the end of the line,
\item [\texttt{ Ctrl-w}] deletes to the start of the word,
\item [\texttt{ Ctrl-y}] pastes text from the clipboard.
\item []
\item [\texttt{ {} Alt-r}] undoes all changes to the line,
\item [\texttt{ Ctrl-r}] searches incrementally up the history,
\item [\texttt{Ctrl-xe}] invokes an editor to write commands,
\item [\texttt{ {} Alt+.}] inserts the last argument of last command.
\end{compactenum}



\section{Shell style guide}
The following notes are meant to be summary of a style guide written by Paul Armstrong and too many more to mention (revision 1.26).

Bash is the only shell scripting language permitted for executables.
Bash should only be used for simple wrapper scripts or small utilities.

Executables should have no extension, libraries must have a \texttt{.sh} extension and should not be executable.
SUID and SGID are \emph{forbidden} on shell scripts.

All error messages should go to \texttt{STDERR},
a function to print out error messages along with other status information is recommended:
\begin{minted}{bash}
err () {
  echo "[$(date +%Y-%m-%d\ %T)]: $@" >&2
}
\end{minted}

\textbf{Comments}.
Start each file with a description of its contents.
Any function that is in a library or not both obvious and short, must be commented.
Comment tricky, interesting or important parts of code.
Use TODO comments for temporary or good enough but not perfect code and short-term solutions.

The following are required for any new code:
\begin{compactenum}
\item Indent 2 spaces, no tabs.
\item Maximum line length is 80 characters.
\item Long pipelines should be split one per line. % if they don't all fit on one line.
% \item Put \texttt{; do} and \texttt{; then} on the same line as the \texttt{while}, \texttt{for} or \texttt{if}.
\item Indent case alternatives by 2 spaces.
\item Always quote strings containing variables, command substitutions or spaces. %or shell meta characters. % , unless unquoted expansion is required. 
\end{compactenum}

Use quotes rather than filler characters if possible.
Use an explicit path when doing wildcard expansion of filenames.
Avoid \texttt{eval}.
Use process substitution or for loops in preference to piping to while.
Finally, \texttt{[[ ... ]]} is preferred over \texttt{[} and \texttt{test}.

\textbf{Naming conventions}.
Function and variable names should be lower case, with underscore to separate words.
Constants and environment variable names should be all caps, declared at the top of the file.
Use \texttt{readonly} or \texttt{declare -r} to ensure they're read only.
Declare function specific variables with \texttt{local}.
A function called main is required for scripts long enough to contain at least one other function.

\subsection{Other useful resources}
If you want to understand Bash better, consider one of the following webpages:
\begin{compactenum}
\item \url{http://wiki.bash-hackers.org}
\item \url{http://mywiki.wooledge.org/BashGuide}
\end{compactenum}

\textbf{Avoid} guides by TLDP (The Linux Documentation Project) at all costs.


\vfill\null\columnbreak


\subsection{sed -- stream editor}
The default delimiters is \texttt{/}, but any other will work too.
Some sed commmands:
\begin{enumx}
\item to substitute (s) strings globally (g):\\
\texttt{sed 's/foo/bar/g'}
\item to substitute only in lines containing \texttt{baz}:\\
\texttt{sed '/baz/s/foo/bar/g'}
\item to substitute only in lines without \texttt{baz}:\\
\texttt{sed '/baz/!s/foo/bar/g'}
\item to use regular expressions:\\
\texttt{sed 's/[0-9]\textbackslash+/(\&)/g'}
\item to delete the first 10 lines:\\
\texttt{sed '1-10d'}
\item to delete the last line:\\
\texttt{sed '\$d'}
\item to edit files in place:\\
\texttt{sed -i 's/foo/bar/g' *.txt}
\item [--] \texttt{n} suppresses automatic printing of pattern space:
\item \texttt{sed -n '45,50p'} \dotfill prints lines 45th to 50th.
\item \texttt{sed -n 'FLEX.\textbackslash\{65\textbackslash\}/p'} \dotfill prints lines of 65 characters or more
\item \texttt{sed -n 'FLEX.\textbackslash\{65\textbackslash\}/!p'} \dotfill prints lines of 64 characters or less
\end{enumx}

\subsection{awk -- Aho, Weinberger, Kernighan}
Awk is a pseudo-C interpretor.
General form of its code:
\begin{minted}[linenos, numbersep=3pt, frame=lines, framesep=1mm]{bash}
BEGIN {initialization}
search pattern {actions}
END {final actions}
\end{minted}

Examples of search patterns:
\begin{enumx}
	\item \texttt{/word[0+9]+/}: regular expressions,
	\item \texttt{!/word[0+9]+/}: negations of these,
	\item \texttt{\$1} \char`\~\, \texttt{/a/}: matches or does not (!\char`\~) lines with a,
	\item \texttt{length(\$0) > 18}.
\end{enumx} 

Important variables:
\begin{enumx}
\item \textbf{FS}: field separator (tab and space by default),
\item \textbf{OFS}: output field separator,
\item \textbf{RS}: record separator (new line),
\item \textbf{NR}: number of the current record,
\item \textbf{NF}: number of fields in the current record. 
\end{enumx}


\section{Regular expressions}
\begin{itemx}
\item POSIX character classes:
\begin{itemx}
\item \texttt{[:alnum:]} $=$ \texttt{[a-zA-Z0-9]}
\item \texttt{[:alpha:]} $=$ \texttt{[a-zA-Z]}
\item \texttt{[:ascii:]} $=$ \texttt{[\textbackslash{}x00-\textbackslash{}x7F]}
\item \texttt{[:blank:]} $=$ \texttt{[ \textbackslash{}t]}
\item \texttt{[:cntrl:]} $=$ \texttt{[\textbackslash{}x00-\textbackslash{}x1F\textbackslash{}x7F]}
\item \texttt{[:digit:]} $=$ \texttt{[0-9]}
\item \texttt{[:graph:]} $=$ \texttt{[\textbackslash{}x21-\textbackslash{}x7E]}
\item \texttt{[:lower:]} $=$ \texttt{[a-z]}
\item \texttt{[:print:]} $=$ \texttt{[\textbackslash{}x20-\textbackslash{}x7E]}
%\item \texttt{[:punct:]} $=$ \texttt{	Punctuation and symbols.	[!"\#$%&'()*+,\-./:;<=>?@\[\\\]^_`{|}~]	\p{P}		\p{Punct}
\item \texttt{[:space:]} $=$ \texttt{[ \textbackslash{}t\textbackslash{}r\textbackslash{}n\textbackslash{}v\textbackslash{}f]}
\item \texttt{[:word:]} $=$ \texttt{[A-Za-z0-9\_]}
\item \texttt{[:xdigit:]} $=$ \texttt{[A-Fa-f0-9]}
\end{itemx}
\item Repetitions:
\begin{itemx}
\item \texttt{*}: $0$ or more, \texttt{+}: $1$ or more, \texttt{?}: $0$ or $1$,
\item \texttt{\{a, b\}}: at least $a$, at most $b$.
\end{itemx}
\item Anchors:
\begin{itemx}
\item \texttt{\textasciicircum}: start of line
\item \texttt{\$}: end of line
\item \texttt{\textbackslash{}<}: start of word
\item \texttt{\textbackslash{}>}: end of word
\end{itemx}
\end{itemx}


\end{multicols*}

\newpage

\begin{multicols*}{3}
\section{Unix utilities and shell builtins}
% Based on https://en.wikipedia.org/wiki/Template:Unix_commands

\subsection{File system}
\begin{enumx}
	\item [\cmd] \textbf{cat} concatenates and prints files:
	\item [\texttt{A}] shows all nonprinting characters,
	\item [\texttt{b}] numbers nonempty output lines,
	\item [\texttt{n}] numbers all output lines,
	\item [\texttt{s}] suppresses repeated empty output lines.
    % tac -r -s 'x|[^x]'
	\item [\cmd] \textbf{tac} does the same in reverse.
	\item [\cmd] \textbf{rev} reverses lines characterwise.
\end{enumx}

\begin{enumx}
	\item [\cmdblack] \textbf{chgrp} changes group ownership.
	
	\item [\cmdblack] \textbf{chmod} changes permissions of a file:
	\item [\texttt{ugoa}] permissions of the owner, group, other/all users,
	\item [\texttt{+-=}] adds, removes or sets selected file mode bits,
	\item [\texttt{rwx}] selects file mode bits: read/write/execute (4/2/1).
	
	\item [\cmdblack] \textbf{chown} changes owner of a file.
	
	\item [\cmd] \textbf{umask} sets file mode creation mask.

	\item [\cmdblack] \textbf{touch} changes file timestamps:
	\item [\texttt{a}] only the access time,
	\item [\texttt{m}] only the modification time,
	\item [\texttt{t}] uses custom stamp instead of current time,
	\item [\texttt{c}] does not create files.
\end{enumx}

\begin{enumx}
	\item [\cmd] \textbf{shasum} prints or checks SHA message digests:
	\item [\texttt{a}] algorithm: 1, 224, 256, 384, 512, 512224 or 512256,
	\item [\texttt{b}] reads in binary mode,
	\item [\texttt{c}] checks SHA sums read from the ,,files''.

	\item [\cmd] See also \textbf{cksum} (CRC checksums) and \textbf{md5sum}.
\end{enumx}

\begin{enumx}
	\item [\cmd] \textbf{dd} converts and copies a file:
	\item [\texttt{if=}] reads from a file instead of standard input,
	\item [\texttt{of=}] writes to a file insteaed of standard output,
	\item [\texttt{bs=}] up to ,,bytes'' bytes at a time,
	\item [\texttt{count=}] copies only ,,n'' input blocks.
\end{enumx}

\begin{enumx}
	\item [\cmdblack] \textbf{cp} copies files and directories:
%	\item [\texttt{a}] never follows symlinks, preserves all attributes,
	\item [\texttt{b}] makes a backup of each existing destination file,
	\item [\texttt{f}] removes an existing destination file if needed,
	\item [\texttt{i}] prompts before overwrite,
	\item [\texttt{n}] does not overwrite existing files,
	\item [\texttt{L}] always follows symlinks in ,,source'',
	\item [\texttt{P}] never follows symlinks in ,,source'',
	\item [\texttt{r}] copies directories recursively,
	\item [\texttt{s}] makes symbolic links instead,
	\item [\texttt{l}] hard links files instead,
	\item [\texttt{t}] copies all ,,source'' arguments into ,,directory'',
	\item [\texttt{T}] treats ,,destination'' as a normal file,
	\item [\texttt{u}] copies only newer source files,
	\item [\texttt{v}] explains what is being done.
	
	\item [\cmdblack] \textbf{mv} moves (renames) files:
	\item [\texttt{b}] makes a backup of each existing destination file,
	\item [\texttt{i}] prompts before overwriting,
	\item [\texttt{f}] does not prompt before overwriting,
	\item [\texttt{n}] does not overwrite existing destination files.
	\item [\texttt{t}] moves all ,,source'' arguments into ,,directory'',
	\item [\texttt{T}] treats ,,destination'' as a normal file,
	\item [\texttt{u}] moves only newer source files,
	\item [\texttt{v}] explains what is being done.
	
	\item [\cmdblack] \textbf{rm} removes files or directories:
	\item [\texttt{f}] never prompts,
	\item [\texttt{i}] always prompts,
	\item [\texttt{r}] removes directories and their contents.

	\item [\cmd] See also \textbf{rmdir} (directories removal).

	\item [\cmd] \textbf{mkdir} makes directories (\texttt{mkdir p}: with parents as needed, no error if existing).
\end{enumx}

\begin{enumx}
	\item [\cmd] \textbf{df} reports file system disk space usage:
	\item [\texttt{h}] prints size in powers of 1024,
	\item [\texttt{i}] list inode information instead of block usage,
	\item [\texttt{t}] limits listing to file systems of given type,
	\item [\texttt{x}] limits listing to file systems not of given type,
	\item [\texttt{T}] prints file systems types.
	
	\item [\cmd] \textbf{du} estimates file space usage:
	\item [\texttt{a}] writes counts for all files, not just directories,
	\item [\texttt{c}] produces a grand total,
	\item [\texttt{d}] the depth at which summing should occur,
	\item [\texttt{h}] prints sizes in human readable format,
	\item [\texttt{s}] diplays only a total.
\end{enumx}


\begin{enumx}
	\item [\cmd] \textbf{file} determines file type.
\end{enumx}

\begin{enumx}
	\item [\cmd] \textbf{fsck} checks and repairs a Linux filesystem:
	\item [\texttt{a}] automatically repairs (without any question!),
	\item [\texttt{t}] specifies the type(s) of filesystem to be checked,
	\item [\texttt{A}] tries to check all filesystems in one run,
	\item [\texttt{M}] skips mounted filesystems,
	\item [\texttt{R}] skips the root filesystem.
\end{enumx}

\begin{enumx}
	\item [\cmd] \textbf{ln} makes hard links between files
	(not directories; only in the same file system):
	\item [\texttt{s}]  makes symbolic links instead.
\end{enumx}

\begin{enumx}
	\item [\cmd] \textbf{ls} lists directory contents:
	\item [\texttt{a}] does not ignore entries starting with dot, 
	\item [\texttt{F}] appends indicator to entries, 
	\item [\texttt{h}] prints human readable sizes, 
	\item [\texttt{i}] prints the index number of each file, 
	\item [\texttt{l}] prints permissions, number of hard links, owner, group, size, last-modified date as well, 
	\item [\texttt{r}] reverses order while sorting,
	\item [\texttt{R}] lists subdirectories recursively, 
	\item [\texttt{S}] sorts by file size (largest first), 
	\item [\texttt{t}] sorts by modification time (newest first), 
	\item [\cmd] \textbf{tree} folds lower case to upper case characters.
\end{enumx}

\begin{enumx}
	\item [\cmd] \textbf{mount} mounts a filesystem.
\end{enumx}

\begin{enumx}
	\item [\cmd] \textbf{pwd} prints name of current directory.
\end{enumx}

\begin{enumx}
	\item [\cmd] \textbf{split} splits a file into pieces:
	\item [\texttt{a}] generates suffixes of length ,,n'' (default 2),
	\item [\texttt{b}] puts ,,size'' bytes per output file,
	\item [\texttt{d}] uses numeric (not alphabetic) suffixes,
	\item [\texttt{l}] puts ,,number'' lines/records per output file,
	\item [\texttt{n}] generates ,,chunks'' output files.
\end{enumx}

\begin{enumx}
	\item [\cmd] \textbf{tar} stores and extracts files from a disk archive:
	\item [\texttt{c}] creates a new archive,
	\item [\texttt{x}] extracts files,
	\item [\texttt{t}] lists the contents of an archive,
	\item [\texttt{v}] verbosely lists files processed,
	\item [\texttt{j}] bzip2 compression,
	\item [\texttt{z}] uses zip/gzip (gz compression),
	\item [\texttt{f}] uses archive file or device (???),
	\item [\texttt{k}] does not replace existing files when extracting.
\end{enumx}

\begin{enumx}
	\item [\cmd] \textbf{tee} (named after the T-splitter used in plumbing) duplicates pipe content:
	\item [\texttt{a}] appends to the given files, does not overwrite,
	\item [\texttt{i}] ignores interrupts.
\end{enumx}

\begin{enumx}
	\item [\cmd] Missing: \textbf{cmp}, \textbf{fuser}, \textbf{pax}, \textbf{type}.
\end{enumx}


\subsection{Processes}
\begin{enumx}
	\item [\cmd] \textbf{chroot} changes the root directory 
	for the current running process and their children.
\end{enumx}

\begin{enumx}
	\item [\cmd] \textbf{at} schedules commands to be executed once, 
	at a particular time in the future: it accepts times of the form 
	\texttt{HH:MM}, \texttt{midnight}, \texttt{noon} or \texttt{teatime}; 
	\texttt{MMDD[CC]YY}, \texttt{MM/DD/[CC]YY}, \texttt{DD.MM.[CC]YY} or 
	\texttt{[CC]YY-MM-DD} (the specification of a date 
	must follow the specification of the time of day).
	You can also give times like \texttt{now + 3 hours}.
\end{enumx}

\begin{enumx}
	\item [\cmd] \textbf{bg} resumes suspended jobs in the background.
	\item [\cmd] \textbf{fg} resumes suspended jobs in the foreground.
	\item [\cmd] \textbf{jobs} lists the active jobs.
	\item [\cmd] \textbf{command \&} runs command in the background.
	% || versus && (error versus success)
\end{enumx}

\begin{enumx}
	\item [\cmd] \textbf{cron}: a daemon executing scheduled commands.
	\item [\cmd] \textbf{crontab} maintain individual users' crontab files.
\end{enumx}

\begin{enumx}
	\item [\cmd] \textbf{kill} sends a \texttt{TERM} signal to a process.
	\item [\cmd] \textbf{killall} kills processes by name.
\end{enumx}

\begin{enumx}
	\item [\cmd] \textbf{ps} reports a snapshot of the current processes.
	\item [\texttt{a}] lifts the ,,only yourself'' restriction,
	\item [\texttt{-e}] selects all processes,
	\item [\texttt{u}] displays user-oriented format,
	\item [\texttt{x}] lifts the ,,must have a tty'' restriction. 
	\item [\cmd] \textbf{pstree} displays a tree of processes.
\end{enumx}

\begin{enumx}
	\item [\cmd] \textbf{nice} changes process priority.
\end{enumx}

\begin{enumx}
	\item [\cmd] \textbf{pgrep}, \textbf{pkill} looks up or signals 
processes based on name and other attributes.
\end{enumx}

\begin{enumx}
	\item [\cmd] \textbf{time} runs programs and summarizes system resource usage. 
\end{enumx}

\begin{enumx}
	\item [\cmd] \textbf{top} displays linux processes.
\end{enumx}


\subsection{User environment}
\begin{compactenum}
	\item [\cmdvar] \textbf{clear} clears the terminal screen.
	\item [\cmdvar] \textbf{env} runs programs in modified environment.
	\item [\cmdvar] \textbf{exit} terminates the calling process.
	\item [\cmdvar] \textbf{finger} looks up user information.
	\item [\cmdvar] \textbf{history}  displays the history list. %with line numbers.
%	\item [\cmdvar] \textbf{logname} prints user's login name.
	\item [\cmdvar] \textbf{mesg} displays messages from other users.
\end{compactenum}

\begin{compactenum}
	\item [\cmdvar] \textbf{passwd} changes user password:
	\item [\texttt{d}] deletes (empties) an account's password,
	\item [\texttt{e}] expires an account's password,
	\item [\texttt{n}] minimum days to change password,
	\item [\texttt{w}] warning days before password expire,
	\item [\texttt{x}] maximum days a password remains valid.
	\item [\cmdvar] \textbf{pwgen} generate pronounceable passwords:
	\item [\texttt{s}] generates hard to memorize passwords,
	\item [\texttt{y}] includes special characters,
	\item [\texttt{n}] includes numbers,
	%\item [\texttt{N}] generates ``num'' passwords
\end{compactenum}

\begin{compactenum}
        \item [\cmdutil] \textbf{su} changes user ID or becomes superuser.
        \item [\cmdvar] \textbf{sudo} executes a command as superuser:
        \item [\texttt{u}] as a different user.
\end{compactenum}

\begin{compactenum}
	\item [\cmdvar] \textbf{hostname} shows/sets the host name:
	\item [\texttt{i}] displays the network address.
	\item [\cmdcore] \textbf{uname} prints system information:
	\item [\texttt{a}] all information, in the following order:
	\item [\texttt{s}] the kernel name,
	\item [\texttt{n}] the network node hostname,
	\item [\texttt{r}] the kernel release,
	\item [\texttt{v}] the kernel version,
	\item [\texttt{m}] the machine hardware name,
	\item [\texttt{p}] the processor type,
	\item [\texttt{i}] the hardware platform,
	\item [\texttt{o}] the operating system.
\end{compactenum}

\begin{compactenum}
	\item [\cmdcore] \textbf{uptime}: how long has system been running?
\end{compactenum}

\begin{compactenum}
	\item [\cmdutil] \textbf{wall} writes a message to all users,
	\item [\cmdvar] \textbf{write} sends a message to another user. 
\end{compactenum}

\begin{compactenum}
	\item [\cmdvar] \textbf{pinky} is a lightweight version of finger.
	\item [\cmdvar] \textbf{who} shows who is logged on,
	\item [\cmdvar] \textbf{w} does the same, shows what they are doing,
	\item [\cmdvar] \textbf{whoami} prints effective userid.
\end{compactenum}

\vfill\null
\columnbreak


\subsection{Text processing}
\renewcommand\theFancyVerbLine{\normalsize\arabic{FancyVerbLine}}

\begin{compactenum}
\item [\cmdvar] \textbf{awk}, \textbf{grep} and \textbf{sed} have been described earlier.
\end{compactenum}

\begin{compactenum}
	\item [\cmdcore] \textbf{comm} compares two sorted files line by line.
	\item [\cmdcore] \textbf{shuf} generates random permutations:
	\item [\texttt{e}] treats each ``arg'' as an input line,
	\item [\texttt{i}] treats each number .. through .. as an input line, 
	\item [\texttt{n}] outputs at most ``count'' lines,
	\item [\texttt{r}] output lines can be repeated (with \texttt{-n}).
	\item [\cmdcore] \textbf{sort} sorts lines of text files:
	\item [\texttt{c}] checks for sorted input,
	\item [\texttt{f}] folds lower case to upper case characters,
	\item [\texttt{g}] compares general numerical values,
	\item [\texttt{h}] compares human readable numbers,
	\item [\texttt{k}] sorts via a key,
	\item [\texttt{n}] compares string numerical values,
	\item [\texttt{r}] reverses the results,
	\item [\texttt{s}] stabilizes the sort.
	\item [\cmdcore] \textbf{tsort} performs topological sort.
	\item [\cmdcore] \textbf{uniq} omits repeated lines:
	\item [\texttt{c}] prefixes lines by the number of occurences,
	\item [\texttt{d}] only prints duplicate lines, one for each group,
	\item [\texttt{f}] avoids comparing first fields,
	\item [\texttt{i}] ignores differences in case,
	\item [\texttt{s}] avoids comparing first characters,
	\item [\texttt{w}] compares no more than $n$ characters.
\end{compactenum}

\begin{compactenum}
	\item [\cmdcore] \textbf{cut} prints selected parts of lines:
	\item [] \texttt{-}\texttt{-}\texttt{complement} complements the selection,
	\item [\texttt{c}] selects only these characters,
	\item [\texttt{d}] uses ``delim'' instead of Tab for field delimeter,
	\item [\texttt{f}] selects only these fields,
	\item [\texttt{s}] does not print lines not containing delimeters.
	\item [\cmdcore] \textbf{join} joins lines of two files on a common field.
	\item [\cmdcore] \textbf{paste} merges lines of files.
	\item [\texttt{d}] reuses characters from ``list'' instead of tabs,
	\item [\texttt{s}] pastes one file at a time, not in parallel.
	\item [\cmdcore] \textbf{tr} translates or deletes characters:
	% \item \texttt{tr abc xyz} changes \texttt{a} to \texttt{x}, $\ldots$,
	\item [c] uses the complement of ``set1'',
	\item [d] deletes characters, does not translate,
	\item [s] replaces each sequence of a repeated character that is listed 
	in the last specified ``set'' with a single occurrence of that character.
\end{compactenum}

\begin{compactenum}
	\item [\cmdvar] \textbf{diff} compares files line by line:
	\item [\texttt{y}] outputs in two columns,
	\item [\texttt{i}] ignores case differences,
	\item [\texttt{w}] ignores all white space.
	% E Z b B
\end{compactenum}

\begin{compactenum}
	\item [\cmdvar] \textbf{fmt} is a simple optimal text formatter, 
	\item [\cmdvar] \textbf{fold} wraps each line to fit in specified width.
\end{compactenum}

\begin{compactenum}
	\item [\cmdcore] \textbf{head} outputs the first (last) part of files:
	\item [\texttt{c}] the first ``num'' bytes,
	\item [\texttt{n}] the first ``num'' lines,
	\item [\cmdcore] \textbf{tail} the last ``num'' bytes:
	\item [\texttt{c}] the last ``num'' bytes,
	\item [\texttt{n}] the last ``num'' lines,
	\item [\texttt{f}] outputs appended data as the file grows,
	\item [\texttt{s}] sleeps for ``n'' seconds between iterations. 
	\item [\cmdcore] \textbf{split} splits a file into pieces:
	\item [\texttt{a}] generates suffixes of length ``n'' (default 2),
	\item [\texttt{b}] puts ``size'' bytes per output file,
	\item [\texttt{d}] uses numeric (not alphabetic) suffixes,
	\item [\texttt{l}] puts ``number'' lines/records per output file,
	\item [\texttt{n}] generates ``chunks'' output files.
	\item [\cmdcore] See also: \textbf{csplit}.
\end{compactenum}

\begin{compactenum}
	\item [\cmdvar] \textbf{more} pages text too large to fit on one screen, allows scrolling down, not up (deprecated!).
	\item [\cmdvar] \textbf{less} is an enhanced version of more:
	\item [\texttt{+F}] monitors the tail of a file which is growing.
\end{compactenum}

\begin{compactenum}
	\item [\cmdvar] \textbf{vim} is an advanced text editor, 
	too complex to be explained here.
	See also \textbf{emacs} or \textbf{nano}.
\end{compactenum}

\begin{compactenum}
	\item [\cmdvar] \textbf{xargs} builds and executes command lines:
	\item [\texttt{0}] takes care of filenames with spaces, backslashes.
	\item [\texttt{I}] replaces occurrences of ``string'' with names read from standard input.
\end{compactenum}

\begin{compactenum}
	\item [\cmdvar] \textbf{yes} outputs a string repeatedly until killed.
\end{compactenum}


\subsection{Shell builtins}
\begin{enumx}
	\item [\cmd] \textbf{alias} allows a string to be substituted for a word.
	\item [\cmd] \textbf{cd} changes the shell working directory:
	\item [\texttt{-}] to the previous directory.
	\item [\cmd] \textbf{echo}* displays a line of text:
	\item [\texttt{e}] enables interpretation of backslash escapes,
	\item [\texttt{n}] does not output the trailing newline.
	\item [\cmd] \textbf{test} checks file types and compares values.
	\item [\cmd] \textbf{unset} unsets a shell variable, removing it from memory and the shell's exported environment.
	\item [\cmd] \textbf{wait} waits for process to change state.
\end{enumx}


\subsection{Networking}
\begin{compactenum}
\item [\cmdvar] \textbf{curl} transfers a URL.
\item [\cmdvar] \textbf{wget} is a non-interactive network downloader.
\item [\texttt{A}, \texttt{R}] specifies lists 	of file suffixes or 
	patterns (when wildcard characters appear) to accept or reject,
\item [\texttt{b}] goes to background immediately after startup,
\item [\texttt{c}] continues getting a partially-downloaded file,
\item [\texttt{m}] turns on options suitable for mirroring: 
	infinite recursion and time-stamping,
\item [\texttt{np}] does not ever ascend to the
	parent directory when retrieving recursively,
\item [\texttt{U}] identifies as ``agent-string'' to the HTTP server.
\item [\texttt{w}] waits the specified number of seconds 
	between the retrievals (see also \texttt{--random-wait}).
\end{compactenum}

\begin{compactenum}
\item [\cmdvar] \textbf{rlogin} starts a terminal session on a remote host.
\item [\cmdvar] \textbf{ssh} is an OpenSSH SSH client (remote login program).
\item [\texttt{D}] specifies a local ''dynamic'' application-level port forwarding,
\item [\texttt{p}] selects a port to connect to on the remote host,
\item [\texttt{X}] enables X11 forwarding.
\end{compactenum}

\begin{compactenum}
\item [\cmdvar] \textbf{dig} interrogates DNS name servers.                        
\item [\texttt{x}] performs a simplified reverse lookup. 
\item [\cmdvar] \textbf{host} is a DNS lookup utility.  
\item [\cmdvar] \textbf{nslookup} is (probably) deprecated! Use \textbf{dig} and \textbf{host}.
\end{compactenum}

\begin{compactenum}
\item [\cmdvar] \textbf{ifconfig} configures a network interface.   
\item [\cmdvar] \textbf{inetd} is a super-server daemon that provides Internet services.
\item [\cmdvar] \textbf{netcat}: arbitrary TCP and UDP connections and listens.
\item [\cmdvar] \textbf{netstat} prints network connections, routing tables, 
interface statistics, masquerade connections, and multicast memberships.
\item [\cmdvar] \textbf{ping} tests the reachability of a host 
on an IP network by sending ICMP ECHO\_REQUEST:
\item [\texttt{c}] stops after sending ``count'' packets,
\item [\texttt{n}] numeric output only, 
	avoids to lookup symbolic names for host addresses. 
\item [\cmdvar] \textbf{rdate} sets the system's date from a remote host.
\item [\cmdvar] \textbf{rsync} copies files fast (remote or local):
\item [\texttt{a}] in archive mode, equivalent to:
\item [\texttt{g}] preserves group,
\item [\texttt{o}] preserves owner (super-user only)
\item [\texttt{p}] preserves permissions,
\item [\texttt{t}] preserves modification times,
\item [\texttt{l}] copies symlinks as symlinks,
\item [\texttt{b}] make backups, 
\item [\texttt{c}] skip based on checksum, 
\item [\texttt{n}] performs a dry run without changes made, 
\item [\texttt{r}] resursively, 
\item [\texttt{u}] skip newer files on the receiver, 
\item [\texttt{v}] increases verbosity, 
\item [\texttt{z}] compresses file data during the transfer,
\item [\texttt{}] \texttt{----delete} deletes extraneous files from dest dirs.
\item [\cmdvar] \textbf{route} shows and manipulates the IP routing table.
\item [\cmdvar] \textbf{traceroute} is a computer network diagnostic tool for 
displaying the route (path) and measuring transit delays of 
\end{compactenum}


\subsection{Searching}
\begin{enumx}
\item [\cmd] \textbf{find} searches for files in a directory hierarchy.
\item [\cmd] \textbf{grep} prints lines matching a pattern.
\item [\cmd] \textbf{locate} finds files by names.
\item [\cmd] \textbf{whatis} displays one-line manual page description.
\item [\cmd] \textbf{whereis} locates the binary, source, 
and manual page files for a command.
\end{enumx}


\subsection{Hardware}
\begin{enumx}
\item [\cmd] \textbf{dmesg} prints/controls the kernel ring buffer.
\item [\cmd] \textbf{lsblk} lists block devices.
\item [\cmd] \textbf{lsof} lists open files.
\item [\cmd] \textbf{lsusb} listsq USB devices.
\end{enumx}


\subsection{For programmers}
\begin{enumx}
	\item [\cmdblack] \textbf{g++} compiles, assembles and links C++ files:
	\item [\texttt{o}] writes the build output to a file named \ldots
\end{enumx}



\subsection{Miscellaneous}
\begin{compactenum}
\item [\cmdvar] \textbf{bc} is an arbitrary precision calculator language.
\item \texttt{echo 'obase=16;255' | bc} prints \texttt{FF},
\item \texttt{echo 'ibase=2;obase=A;10' | bc} prints \texttt{2},
\item \texttt{scale=10} (after \texttt{bc -l}) sets working precision.
\item [\cmdvar] \textbf{dc} is a reverse-polish desk calculator.
One of the oldest Unix utilities, 
predating even the invention of the C programming language.
\item [\cmdutil] \textbf{cal}, \textbf{ncal} displays a calendar.
\item [\texttt{e}] displays date of Easter,
\item [\texttt{j}] displays Julian days,
\item [\texttt{m}] displays the specified month,
\item [\texttt{w}] prints the numbers of the weeks,
\item [\texttt{y}] displays a calendar for the specified year,
\item [\texttt{3}] displays the previous, current and next month.
\item [\cmdvar] \textbf{date} prints or set the system date and time.
% \textbf{expr}
%\item [\cmdvar] \textbf{lp} prints files.
%\item [\cmdvar] \textbf{od} dumps files in octal.
% hexdump -C, xxd
\item [\cmdcore] \textbf{seq} prints a sequence of numbers:
\item [\texttt{w}] equalizes width by padding with leading zeroes.
\item [\cmdcore] \textbf{sleep} delays for a specified amount of time.
\item [\cmdvar] \textbf{true}, \textbf{false} does nothing, (un)successfully.
\end{compactenum}
\end{multicols*}
\end{document}

Todo:
\textbf{gp} invokes the PARI/GP calculator.
\textbf{pdflatex} runs the pdfTeX typesetter.

\textbf{apropos} searches the manual page names and descriptions.
\textbf{man} is an interface to the online reference manuals.

\textbf{ghci} is the Glasgow Haskell Compiler.
\textbf{ipython} is an interactive Python shell, see also
\textbf{python} and \textbf{python3}.
\textbf{gcc} is a C and C++ compiler.

nc - ?
